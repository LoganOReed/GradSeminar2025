%! TEX program = xelatex
%! TEX root = **/000-main.tex
% vim: spell spelllang=en:

\documentclass{beamer}

\usepackage{blindtext}
\usepackage{algorithm}
\usepackage{algpseudocode}
\usepackage{import}
\usepackage{pdfpages}
\usepackage{transparent}
\usepackage{xcolor}



% \pdfsuppresswarningpagegroup=1



\usetheme[nopixelitem]{Execushares}

\title{Krylov Subspace Methods}
\subtitle{Using Projection Methods to Approximate Inverses}
\author{Logan Reed}
\date{October 3, 2025}

\setcounter{showSlideNumbers}{1}

\newcommand{\incfig}[2][1]{%
    \def\svgwidth{#1\columnwidth}
    \import{./figures/}{#2.pdf_tex}
}

\begin{document}
	\setcounter{showProgressBar}{0}
	\setcounter{showSlideNumbers}{0}

	\frame{\titlepage}

	\begin{frame}
		\frametitle{Contents} \begin{enumerate}
			\item Introduction \\ \textcolor{ExecusharesGrey}{\footnotesize\hspace{1em} Problems to solve and the two major methods}
			\item Projection Methods  \\ \textcolor{ExecusharesGrey}{\footnotesize\hspace{1em} A canonical way of explaining iterative methods}
			\item Krylov Subspace Methods \\ \textcolor{ExecusharesGrey}{\footnotesize\hspace{1em} CG, GMRES, and connection to minimal polynomials}
			\item Implementation Details \\ \textcolor{ExecusharesGrey}{\footnotesize\hspace{1em} Preconditioning, restarting, and picking the right basis}
			\item Numerical Results \\ \textcolor{ExecusharesGrey}{\footnotesize\hspace{1em} Comparing different methods and augmentations}
			\item Conclusion \\ \textcolor{ExecusharesGrey}{\footnotesize\hspace{1em} How to choose which method to use}
		\end{enumerate}
	\end{frame}

	\setcounter{framenumber}{0}
	\setcounter{showProgressBar}{1}
	\setcounter{showSlideNumbers}{1}
	\section{Introduction}
    \begin{frame}
			\frametitle{The Problem}
      We wish to solve large invertible systems of linear equations of the form
      \[
      Ax = b
      .\] 
      The difficulty is determined by
      \begin{itemize}
        \item Size. Current problems have $\text{dim}\left( A \right) > 10000$
        \item Sparsity. The structure of zero entries can be used to improve computation time
        \item Condition. $\kappa \left( A \right) = \left\| A \right\| \left\| A^{-1} \right\| > > 1$ will reduce convergence rate.
      \end{itemize}
    \end{frame}

    \begin{frame}
      \frametitle{Direct Methods}
      Direct Methods give the exact solution to $Ax = b$ in a finite number of steps in the absence of roundoff error.
      They are all variations of Gaussian elimination.
       \begin{itemize}
         \item Pros
           \begin{itemize}
             \item Guaranteed solution in a fixed amount of time
             \item Stability is known a priori
             \item Works for a Matrix without known properties
           \end{itemize}
         \item Cons
           \begin{itemize}
             \item Roundoff Error cause compounding accuracy and stability issues
             \item Sparse matrices are treated the same as dense matrices
             \item The entire matrix must be stored in memory
           \end{itemize}
      \end{itemize}
    \end{frame}

    \begin{frame}
      \frametitle{Iterative Methods}
      Iterative Methods use successive approximations to obtain more accurate solutions at each step.
      \begin{itemize}
        \item Pros
          \begin{itemize}
            \item Leverages properties of the matrix
            \item Doesn't need to store $A$ directly, stores matvecs.
            \item Especially effective when $A$ is sparse, e.g. from solving PDEs.
          \end{itemize}
        \item Cons
          \begin{itemize}
            \item No guarantee of convergence in a given number of steps
            \item Not competitive with Direct Methods for dense $A$
            \item Knowledge of  $A$ is needed for competitive runtime
          \end{itemize}
      \end{itemize}
    \end{frame}

    \begin{frame}
      \frametitle{Projection Methods}
      Projection Methods are Iterative Methods which find, at each time step, an approximate solution to a problem from a subspace.

      \begin{itemize}
        \item Initial Problem: $Ax = b$
          \begin{itemize}
            \item $A \in \mathbb{R}^{n \times n}$
          \end{itemize}
        \item Projected Problem: $b-A \widetilde{x} \perp \mathcal{L}$
          \begin{itemize}
            \item $\mathcal{K},\mathcal{L} \in \mathbb{R}^{m \times m}$ where $m < < n$
            \item $\widetilde{x} \in \mathcal{K}$
          \end{itemize}
      \end{itemize}
      $m$ degrees of freedom ($\mathcal{K}$) and $m$ constraints ($\mathcal{L}$) gives an $n \times n$ system.
    \end{frame}

    \begin{frame}
      \frametitle{Projection Examples}
      Steepest Descent: 
      \begin{itemize}
        \item $A$ symmetric positive definite (spd)
        \item $\mathcal{K} = \text{span}(r)$ 
        \item $\mathcal{L} = \mathcal{K}$
      \end{itemize}
\begin{algorithm}[H]
\begin{algorithmic}
\For{$i=1,\ldots$}
\State $r \gets b - A x$ 
\State $\alpha \gets \frac{\left( r,r \right)}{\left( Ar,r \right)}$ \Comment{Step Size}
\State $x \gets x + \alpha r$
\EndFor
\end{algorithmic}
\caption{One Dimensional Steepest Descent}
\end{algorithm}
Each step minimizes $\left\| x - x^{*} \right\|_{A}^{2}$ in direction $-\nabla f$.
    \end{frame}

\begin{frame}
% TODO: Type name and press <C-f> to open inkscape
  % NOTE: You also need to start it with space a rn.
\end{frame}

		\begin{frame}
			\frametitle{The Problem}
			\begin{enumerate}
				\item LaTeX is great!
				\item Beamer is easy to use!
          \begin{enumerate}
            \item Subitems
            \item subitems
            \begin{enumerate}
              \item subsubitems weeeee.
            \end{enumerate}
          \end{enumerate}
				\item Why not?
			\end{enumerate}
		\end{frame}

		\begin{frame}
			\frametitle{Why Custom Themes?}
			\begin{itemize}
				\item The default Beamer themes are outdated and visually displeasing
          \begin{itemize}
            \item Subitems
            \item subitems
            \begin{itemize}
              \item subsubitems weeeee.
            \end{itemize}
          \end{itemize}
				\item There aren't many Beamer themes readily available online
				\item Making custom Beamer themes is easy!
			\end{itemize}
		\end{frame}

	\section{Lorem Ipsum}
		\begin{frame}
			\frametitle{Lorem 1}
			\blindtext
		\end{frame}

		\begin{frame}
			\frametitle{Lorem 2}
			\blindtext
		\end{frame}

		\begin{frame}
			\frametitle{Lorem 3}
			\blindtext
		\end{frame}

	\section{Conclusions}
		\begin{frame}
			\frametitle{Closing Thoughts}
			\begin{itemize}
				\item Woo, Beamer!
			\end{itemize}
		\end{frame}
	
	\appendix
	\backupbegin
	  \begin{frame}
	    \frametitle{Backup slide 1}
	    \blindtext
	  \end{frame}
	\backupend

\end{document}
